\documentclass[aspectratio=169]{beamer}
\usepackage[utf8]{inputenc}
\usepackage[french]{babel}
\usepackage{graphicx}
\usepackage{amsmath}
\usepackage{amsfonts}
\usepackage{amssymb}
\usepackage{listings}
\usepackage{xcolor}
\usepackage{tikz}
\usepackage{hyperref}
\usepackage{animate}
\usepackage{transparent}

% Configuration des couleurs modernes
\definecolor{primaryblue}{RGB}{25, 118, 210}
\definecolor{secondaryblue}{RGB}{69, 90, 121}
\definecolor{accentgreen}{RGB}{76, 175, 80}
\definecolor{warningorange}{RGB}{255, 152, 0}
\definecolor{errorred}{RGB}{244, 67, 54}
\definecolor{lightgray}{RGB}{245, 245, 245}
\definecolor{darkgray}{RGB}{33, 33, 33}

% Configuration du thème moderne
\usetheme{Madrid}
\usecolortheme{default}

% Personnalisation du thème
\setbeamercolor{structure}{fg=primaryblue}
\setbeamercolor{title}{fg=white,bg=primaryblue}
\setbeamercolor{frametitle}{fg=white,bg=secondaryblue}
\setbeamercolor{block title}{fg=white,bg=primaryblue}
\setbeamercolor{block body}{fg=darkgray,bg=lightgray}
\setbeamercolor{alerted text}{fg=errorred}
\setbeamercolor{example text}{fg=accentgreen}

% Configuration du footer
\setbeamertemplate{footline}[frame number]
\setbeamertemplate{headline}{}
\setbeamertemplate{navigation symbols}{}

% Personnalisation du footer
\setbeamertemplate{footline}{
  \leavevmode%
  \hbox{%
  \begin{beamercolorbox}[wd=.25\paperwidth,ht=2.25ex,dp=1ex,center]{author in head/foot}%
    \usebeamerfont{author in head/foot}\insertshortauthor~~(\insertshortinstitute)
  \end{beamercolorbox}%
  \begin{beamercolorbox}[wd=.5\paperwidth,ht=2.25ex,dp=1ex,center]{title in head/foot}%
    \usebeamerfont{title in head/foot}\insertshorttitle
  \end{beamercolorbox}%
  \begin{beamercolorbox}[wd=.25\paperwidth,ht=2.25ex,dp=1ex,right]{date in head/foot}%
    \usebeamerfont{date in head/foot}\insertshortdate{}\hfill\insertframenumber{} / \inserttotalframenumber\hspace*{2em} 
  \end{beamercolorbox}}%
  \vskip0pt%
}

% Police moderne
\usepackage[T1]{fontenc}

% Configuration des listes
\lstset{
    language=bash,
    basicstyle=\tiny\ttfamily,
    keywordstyle=\color{blue},
    commentstyle=\color{gray},
    stringstyle=\color{red},
    numbers=left,
    numberstyle=\tiny,
    stepnumber=1,
    numbersep=5pt,
    backgroundcolor=\color{gray!10},
    showspaces=false,
    showstringspaces=false,
    showtabs=false,
    frame=single,
    tabsize=2,
    captionpos=b,
    breaklines=true,
    breakatwhitespace=false,
    escapeinside={\%*}{*)}
}

% Informations du document
\title[OAuth 2.0 avec Keycloak]{\textbf{Projet 2: Implémentation OAuth 2.0 dans une Application Web}}
\subtitle{\large Mise en place d'un serveur d'autorisation avec Keycloak}
\author[Ahmed, Moussa]{Ahmed, Moussa}
\institute[ ENET'COM]{ENET'COM}
\date{\today}

\begin{document}

% Page de titre moderne
\begin{frame}
    \begin{center}
        \vspace{-0.5cm}
        \begin{tikzpicture}
            \node[fill=primaryblue, text=white, minimum width=\textwidth, minimum height=2.5cm, rounded corners=10pt] at (0,0) {
                \Huge \textbf{OAuth 2.0 avec Keycloak}
            };
        \end{tikzpicture}
        
        \vspace{0.8cm}
        \Large \textbf{Projet 2: Implémentation OAuth 2.0 dans une Application Web}
        
        \vspace{0.5cm}
        \large Mise en place d'un serveur d'autorisation avec Keycloak
        
        \vspace{1.2cm}
        \begin{tikzpicture}
            \node[fill=secondaryblue, text=white, minimum width=10cm, minimum height=1.8cm, rounded corners=8pt] at (0,0) {
                \large \textbf{Étudiants:} Ahmed , Moussa\\
                \textbf{ ENET'COM} \quad Enseignant: Mr ZARAI F.
            };
        \end{tikzpicture}
        
        \vspace{0.8cm}
        \large \today
    \end{center}
\end{frame}

% Table des matières moderne
\begin{frame}
    \frametitle{\faList \quad Plan de la Présentation}
    \begin{columns}
        \begin{column}{0.5\textwidth}
            \tableofcontents[sections={1-3}]
        \end{column}
        \begin{column}{0.5\textwidth}
            \tableofcontents[sections={4-6}]
        \end{column}
    \end{columns}
\end{frame}

\section{Introduction à OAuth 2.0}

\begin{frame}
    \frametitle{\faShieldAlt \quad Qu'est-ce qu'OAuth 2.0?}
    \begin{alertblock}{Définition}
        \Large OAuth 2.0 est un \textbf{framework d'autorisation} qui permet à une application d'obtenir un accès limité aux ressources d'un utilisateur sur un service HTTP.
    \end{alertblock}
    
    \vspace{0.5cm}
    \begin{columns}
        \begin{column}{0.5\textwidth}
            \begin{block}{\faExclamationTriangle \quad Problème résolu}
                \begin{itemize}
                    \item \textcolor{errorred}{Éviter le partage de mots de passe}
                    \item \textcolor{primaryblue}{Contrôle granulaire des permissions}
                    \item \textcolor{warningorange}{Révocabilité des accès}
                    \item \textcolor{accentgreen}{Sécurité renforcée}
                \end{itemize}
            \end{block}
        \end{column}
        \begin{column}{0.5\textwidth}
            \begin{block}{\faCheckCircle \quad Avantages}
                \begin{itemize}
                    \item \textcolor{accentgreen}{Standard industriel}
                    \item \textcolor{primaryblue}{Interopérabilité}
                    \item \textcolor{warningorange}{Flexibilité}
                    \item \textcolor{accentgreen}{Sécurité éprouvée}
                \end{itemize}
            \end{block}
        \end{column}
    \end{columns}
\end{frame}

\begin{frame}
    \frametitle{Acteurs OAuth 2.0}
    \begin{center}
        \begin{tikzpicture}[scale=0.8]
            % Resource Owner
            \node[draw, rectangle, fill=primaryblue!20, text=primaryblue, minimum width=3cm, minimum height=1.5cm, rounded corners=8pt] (ro) at (-4,2) {
                \textbf{Resource Owner}\\
                Utilisateur
            };
            
            % Client
            \node[draw, rectangle, fill=accentgreen!20, text=accentgreen, minimum width=3cm, minimum height=1.5cm, rounded corners=8pt] (client) at (-4,-2) {
                \textbf{Client}\\
                Application Web
            };
            
            % Authorization Server
            \node[draw, rectangle, fill=errorred!20, text=errorred, minimum width=3cm, minimum height=1.5cm, rounded corners=8pt] (auth) at (4,2) {
                \textbf{Authorization Server}\\
                Keycloak
            };
            
            % Resource Server
            \node[draw, rectangle, fill=warningorange!20, text=warningorange, minimum width=3cm, minimum height=1.5cm, rounded corners=8pt] (resource) at (4,-2) {
                \textbf{Resource Server}\\
                API Protégée
            };
            
            % Arrows
            \draw[->, thick, primaryblue] (ro) -- (auth) node[midway, above, primaryblue, font=\footnotesize] {1. Autorisation};
            \draw[->, thick, accentgreen] (client) -- (auth) node[midway, right, accentgreen, font=\footnotesize] {2. Demande token};
            \draw[->, thick, errorred] (auth) -- (client) node[midway, right, errorred, font=\footnotesize] {3. Token d'accès};
            \draw[->, thick, warningorange] (client) -- (resource) node[midway, below, warningorange, font=\footnotesize] {4. Accès ressource};
        \end{tikzpicture}
    \end{center}
\end{frame}

\begin{frame}
    \frametitle{Flux d'Autorisation OAuth 2.0}
    \begin{columns}
        \begin{column}{0.6\textwidth}
            \begin{enumerate}
                \item \textcolor{primaryblue}{\textbf{Redirection vers l'autorisation:}} L'application redirige l'utilisateur vers le serveur d'autorisation
                \item \textcolor{accentgreen}{\textbf{Authentification:}} L'utilisateur s'authentifie et autorise l'application
                \item \textcolor{warningorange}{\textbf{Code d'autorisation:}} Le serveur renvoie un code d'autorisation
                \item \textcolor{errorred}{\textbf{Échange de token:}} L'application échange le code contre un token d'accès
                \item \textcolor{primaryblue}{\textbf{Accès aux ressources:}} L'application utilise le token pour accéder aux ressources
            \end{enumerate}
        \end{column}
        \begin{column}{0.4\textwidth}
            \begin{center}
                \begin{tikzpicture}[scale=0.6]
                    \node[fill=primaryblue!20, text=primaryblue, minimum width=3cm, minimum height=0.8cm, rounded corners=5pt] at (0,4) {1. Redirection};
                    \node[fill=accentgreen!20, text=accentgreen, minimum width=3cm, minimum height=0.8cm, rounded corners=5pt] at (0,3) {2. Auth};
                    \node[fill=warningorange!20, text=warningorange, minimum width=3cm, minimum height=0.8cm, rounded corners=5pt] at (0,2) {3. Code};
                    \node[fill=errorred!20, text=errorred, minimum width=3cm, minimum height=0.8cm, rounded corners=5pt] at (0,1) {4. Token};
                    \node[fill=primaryblue!20, text=primaryblue, minimum width=3cm, minimum height=0.8cm, rounded corners=5pt] at (0,0) {5. Accès};
                \end{tikzpicture}
            \end{center}
        \end{column}
    \end{columns}
    
    \vspace{0.5cm}
    \begin{alertblock}{Sécurité}
        Le code d'autorisation est échangé contre le token d'accès côté serveur, jamais côté client.
    \end{alertblock}
\end{frame}

\section{Keycloak: Serveur d'Autorisation}

\begin{frame}
    \frametitle{Présentation de Keycloak}
    \begin{alertblock}{Keycloak}
        \Large Keycloak est un \textbf{serveur d'authentification et d'autorisation} open-source qui implémente les standards OAuth 2.0, OpenID Connect et SAML.
    \end{alertblock}
    
    \vspace{0.5cm}
    \begin{columns}
        \begin{column}{0.5\textwidth}
            \begin{block}{Fonctionnalités}
                \begin{itemize}
                    \item \textcolor{primaryblue}{Authentification unique (SSO)}
                    \item \textcolor{accentgreen}{Gestion des utilisateurs}
                    \item \textcolor{warningorange}{Gestion des rôles et permissions}
                    \item \textcolor{errorred}{Intégration LDAP/AD}
                    \item \textcolor{primaryblue}{Interface d'administration}
                    \item \textcolor{accentgreen}{API REST complète}
                \end{itemize}
            \end{block}
        \end{column}
        \begin{column}{0.5\textwidth}
            \begin{block}{Avantages}
                \begin{itemize}
                    \item \textcolor{accentgreen}{Open source}
                    \item \textcolor{primaryblue}{Standards conformes}
                    \item \textcolor{warningorange}{Haute disponibilité}
                    \item \textcolor{errorred}{Scalabilité}
                    \item \textcolor{primaryblue}{Documentation complète}
                    \item \textcolor{accentgreen}{Communauté active}
                \end{itemize}
            \end{block}
        \end{column}
    \end{columns}
\end{frame}

\begin{frame}
    \frametitle{Architecture Keycloak}
    \begin{center}
        \begin{tikzpicture}[scale=0.6]
            % Keycloak Server - Central
            \node[draw, rectangle, fill=primaryblue!20, text=primaryblue, minimum width=4cm, minimum height=2cm, rounded corners=8pt] (keycloak) at (0,0) {
                \textbf{Keycloak Server}\\
                Serveur d'Autorisation
            };
            
            % Components - Well spaced around
            \node[draw, rectangle, fill=accentgreen!20, text=accentgreen, minimum width=2.5cm, minimum height=1cm, rounded corners=5pt] (auth) at (-4,2) {Auth Server};
            \node[draw, rectangle, fill=warningorange!20, text=warningorange, minimum width=2.5cm, minimum height=1cm, rounded corners=5pt] (token) at (4,2) {Token Manager};
            \node[draw, rectangle, fill=errorred!20, text=errorred, minimum width=2.5cm, minimum height=1cm, rounded corners=5pt] (user) at (-4,-2) {User Management};
            \node[draw, rectangle, fill=primaryblue!20, text=primaryblue, minimum width=2.5cm, minimum height=1cm, rounded corners=5pt] (admin) at (4,-2) {Admin Console};
            
            % Realm Management - Below
            \node[draw, rectangle, fill=accentgreen!20, text=accentgreen, minimum width=3cm, minimum height=1cm, rounded corners=5pt] (realm) at (0,-3.5) {Realm Management};
            
            % Database - Bottom
            \node[draw, rectangle, fill=darkgray!20, text=darkgray, minimum width=3.5cm, minimum height=1cm, rounded corners=5pt] (db) at (0,-5.5) {Base de données PostgreSQL};
            
            % Arrows - Clear connections
            \draw[->, thick, primaryblue] (auth) -- (keycloak);
            \draw[->, thick, warningorange] (token) -- (keycloak);
            \draw[->, thick, errorred] (user) -- (keycloak);
            \draw[->, thick, primaryblue] (admin) -- (keycloak);
            \draw[<->, thick, accentgreen] (keycloak) -- (realm);
            \draw[<->, thick, darkgray] (realm) -- (db);
        \end{tikzpicture}
    \end{center}
\end{frame}

\section{Implémentation Pratique}

\begin{frame}
    \frametitle{Architecture du Projet}
    \begin{center}
        \begin{tikzpicture}[scale=0.6]
            % User - Top center
            \node[draw, circle, fill=primaryblue!20, text=primaryblue, minimum width=2.5cm, minimum height=2.5cm] (user) at (0,5) {
                Utilisateur
            };
            
            % Web App - Left middle
            \node[draw, rectangle, fill=accentgreen!20, text=accentgreen, minimum width=3.5cm, minimum height=1.8cm, rounded corners=8pt] (webapp) at (-4,2) {
                \textbf{Application Web}\\
                Node.js + Express
            };
            
            % Keycloak - Right middle
            \node[draw, rectangle, fill=errorred!20, text=errorred, minimum width=3.5cm, minimum height=1.8cm, rounded corners=8pt] (keycloak) at (4,2) {
                \textbf{Keycloak}\\
                Serveur d'Autorisation
            };
            
            % API Protected Resources - Left bottom
            \node[draw, rectangle, fill=warningorange!20, text=warningorange, minimum width=3.5cm, minimum height=1.8cm, rounded corners=8pt] (api) at (-4,-1) {
                \textbf{API Protégée}\\
                Ressources
            };
            
            % Database - Right bottom
            \node[draw, rectangle, fill=darkgray!20, text=darkgray, minimum width=3.5cm, minimum height=1.8cm, rounded corners=8pt] (db) at (4,-1) {
                \textbf{PostgreSQL}\\
                Base de données
            };
            
            % Arrows
            \draw[->, thick, primaryblue] (user) -- node[left] {1. Login} (webapp);
            \draw[->, thick, accentgreen] (webapp) -- node[above] {2. Auth} (keycloak);
            \draw[->, thick, errorred] (keycloak) -- node[right] {3. Data} (db);
            \draw[->, thick, warningorange] (webapp) -- node[left] {4. API} (api);
        \end{tikzpicture}
    \end{center}
\end{frame}

\begin{frame}[fragile]
    \frametitle{Configuration Docker Compose}
    \begin{lstlisting}[language=bash, caption=Configuration des services]
version: '3.8'

services:
  keycloak:
    image: quay.io/keycloak/keycloak:22.0
    environment:
      KEYCLOAK_ADMIN: admin
      KEYCLOAK_ADMIN_PASSWORD: admin123
      KC_DB: postgres
      KC_DB_URL: jdbc:postgresql://postgres:5432/keycloak
    ports:
      - "8080:8080"
    command: start-dev

  postgres:
    image: postgres:15
    environment:
      POSTGRES_DB: keycloak
      POSTGRES_USER: keycloak
      POSTGRES_PASSWORD: keycloak123

  web-app:
    build: ./web-app
    ports:
      - "3000:3000"
    environment:
      KEYCLOAK_URL: http://keycloak:8080
      KEYCLOAK_REALM: oauth-demo
      KEYCLOAK_CLIENT_ID: web-app-client
    \end{lstlisting}
\end{frame}

\begin{frame}[fragile]
    \frametitle{Configuration OAuth 2.0 dans l'Application}
    \begin{lstlisting}[language=JavaScript, caption=Configuration Passport.js]
const { Issuer, Strategy: OpenIDConnectStrategy } = require('openid-client');

// Découverte du serveur Keycloak
const keycloakIssuer = await Issuer.discover(
  `${KEYCLOAK_URL}/realms/${KEYCLOAK_REALM}`
);

// Configuration du client
const client = new keycloakIssuer.Client({
  client_id: KEYCLOAK_CLIENT_ID,
  client_secret: KEYCLOAK_CLIENT_SECRET,
  redirect_uris: ['http://localhost:3000/auth/callback'],
  response_types: ['code'],
});

// Configuration de la stratégie Passport
passport.use('oidc', new OpenIDConnectStrategy({
  client: client,
  params: { scope: 'openid email profile' }
}, (tokenset, userinfo, done) => {
  return done(null, userinfo);
}));
    \end{lstlisting}
\end{frame}

\begin{frame}[fragile]
    \frametitle{Routes d'Authentification}
    \begin{lstlisting}[language=JavaScript, caption=Routes Express.js]
// Route de connexion
app.get('/login', passport.authenticate('oidc'));

// Callback OAuth 2.0
app.get('/auth/callback', 
  passport.authenticate('oidc', { failureRedirect: '/error' }),
  (req, res) => {
    res.redirect('/dashboard');
  }
);

// Route protégée
app.get('/api/protected', (req, res) => {
  if (!req.isAuthenticated()) {
    return res.status(401).json({ error: 'Unauthorized' });
  }
  
  res.json({
    message: 'This is a protected resource',
    user: req.user,
    timestamp: new Date().toISOString()
  });
});

// Déconnexion
app.get('/logout', (req, res) => {
  req.logout((err) => {
    if (err) console.error('Logout error:', err);
    res.redirect('/');
  });
});
    \end{lstlisting}
\end{frame}

\section{Sécurité et Bonnes Pratiques}

\begin{frame}
    \frametitle{Considérations de Sécurité}
    \begin{block}{Sécurité OAuth 2.0}
        \begin{itemize}
            \item \textbf{HTTPS obligatoire} en production
            \item \textbf{Validation des tokens} côté serveur
            \item \textbf{Expiration des tokens} configurée
            \item \textbf{Refresh tokens} pour la continuité
            \item \textbf{Scopes limités} aux besoins réels
        \end{itemize}
    \end{block}
    
    \begin{alertblock}{Risques à éviter}
        \begin{itemize}
            \item Stockage des tokens côté client
            \item Transmission en HTTP
            \item Tokens avec durée de vie excessive
            \item Scopes trop permissifs
            \item Gestion d'erreurs révélatrice
        \end{itemize}
    \end{alertblock}
\end{frame}

\begin{frame}
    \frametitle{Configuration Sécurisée Keycloak}
    \begin{columns}
        \begin{column}{0.5\textwidth}
            \textbf{Paramètres de sécurité:}
            \begin{itemize}
                \item Durée de vie des tokens
                \item Politique de mots de passe
                \item Authentification à deux facteurs
                \item Audit des connexions
                \item Rate limiting
                \item CORS configuré
            \end{itemize}
        \end{column}
        \begin{column}{0.5\textwidth}
            \textbf{Monitoring:}
            \begin{itemize}
                \item Logs d'authentification
                \item Métriques de performance
                \item Alertes de sécurité
                \item Tableau de bord admin
                \item Rapports d'audit
                \item Intégration SIEM
            \end{itemize}
        \end{column}
    \end{columns}
\end{frame}

\section{Démonstration}

\begin{frame}
    \frametitle{Fonctionnalités Implémentées}
    \begin{block}{Application Web}
        \begin{itemize}
            \item \textbf{Page d'accueil} avec présentation du projet
            \item \textbf{Authentification OAuth 2.0} via Keycloak
            \item \textbf{Dashboard utilisateur} avec informations de session
            \item \textbf{Profil utilisateur} avec détails du token JWT
            \item \textbf{API protégée} démontrant l'utilisation des tokens
            \item \textbf{Gestion des erreurs} et messages utilisateur
        \end{itemize}
    \end{block}
    
    \begin{block}{Interface Utilisateur}
        \begin{itemize}
            \item Design responsive avec Bootstrap 5
            \item Interface en français
            \item Navigation intuitive
            \item Feedback visuel des actions
            \item Gestion des états d'authentification
        \end{itemize}
    \end{block}
\end{frame}

\begin{frame}
    \frametitle{Flux d'Utilisation}
    \begin{enumerate}
        \item \textbf{Accès à l'application} sur \texttt{http://localhost:3000}
        \item \textbf{Connexion} via le bouton "Se connecter avec OAuth 2.0"
        \item \textbf{Redirection vers Keycloak} pour l'authentification
        \item \textbf{Authentification} avec les identifiants de test
        \item \textbf{Retour à l'application} avec session active
        \item \textbf{Accès au dashboard} et aux ressources protégées
        \item \textbf{Consultation du profil} avec détails du token JWT
        \item \textbf{Test de l'API} protégée via interface web
        \item \textbf{Déconnexion} et retour à l'état initial
    \end{enumerate}
\end{frame}

\section{Conclusion}

\begin{frame}
    \frametitle{Points Clés du Projet}
    \begin{block}{Objectifs Atteints}
        \begin{itemize}
            \item \textbf{Compréhension théorique} d'OAuth 2.0 et de ses flux
            \item \textbf{Implémentation pratique} avec Keycloak
            \item \textbf{Application web fonctionnelle} avec authentification
            \item \textbf{Sécurité respectée} selon les bonnes pratiques
            \item \textbf{Documentation complète} du projet
        \end{itemize}
    \end{block}
    
    \begin{block}{Compétences Développées}
        \begin{itemize}
            \item Configuration et administration de Keycloak
            \item Intégration OAuth 2.0 dans une application Node.js
            \item Gestion des sessions et tokens JWT
            \item Développement d'interfaces utilisateur sécurisées
            \item Containerisation avec Docker
        \end{itemize}
    \end{block}
\end{frame}

\begin{frame}
    \frametitle{Perspectives d'Évolution}
    \begin{block}{Améliorations Possibles}
        \begin{itemize}
            \item \textbf{Multi-tenancy} avec plusieurs realms
            \item \textbf{Intégration LDAP/AD} pour l'authentification
            \item \textbf{Authentification à deux facteurs} (2FA)
            \item \textbf{API Gateway} pour la gestion centralisée
            \item \textbf{Monitoring et observabilité} avancés
            \item \textbf{Tests automatisés} et CI/CD
        \end{itemize}
    \end{block}
    
    \begin{block}{Cas d'Usage Réels}
        \begin{itemize}
            \item Applications d'entreprise avec SSO
            \item APIs microservices sécurisées
            \item Portails clients avec authentification fédérée
            \item Applications mobiles avec OAuth 2.0
        \end{itemize}
    \end{block}
\end{frame}

\begin{frame}
    \begin{center}
        \vspace{1cm}
        \begin{tikzpicture}
            \node[fill=primaryblue, text=white, minimum width=\textwidth, minimum height=3cm, rounded corners=10pt] at (0,0) {
                \Huge \textbf{Merci pour votre attention!}
            };
        \end{tikzpicture}
        
               \vspace{0.8cm}
        \begin{tikzpicture}
            \node[fill=warningorange!20, text=warningorange, minimum width=10cm, minimum height=1cm, rounded corners=8pt] at (0,-1) {
                \normalsize \textbf{Projet disp sur:} \url{https://github.com/ahmedabddayme3752/oauth-keycloak-project}
            };
        \end{tikzpicture}
        
        \vspace{0.5cm}
        \begin{tikzpicture}
            \node[fill=lightgray, text=darkgray, minimum width=10cm, minimum height=0.8cm, rounded corners=8pt] at (0,-2) {
                \normalsize \textbf{Références:} OAuth 2.0 RFC 6749, Keycloak Documentation, OpenID Connect
            };
        \end{tikzpicture}
    \end{center}
\end{frame}

\end{document}
