\documentclass[aspectratio=169]{beamer}
\usepackage[utf8]{inputenc}
\usepackage[french]{babel}
\usepackage{graphicx}
\usepackage{amsmath}
\usepackage{amsfonts}
\usepackage{amssymb}
\usepackage{listings}
\usepackage{xcolor}
\usepackage{tikz}
\usepackage{hyperref}

% Configuration des couleurs
\definecolor{keycloakblue}{RGB}{0, 112, 210}
\definecolor{oauthgreen}{RGB}{0, 150, 0}
\definecolor{errorred}{RGB}{220, 20, 60}

% Configuration du thème
\usetheme{Madrid}
\usecolortheme{default}

% Configuration des listes
\lstset{
    language=JavaScript,
    basicstyle=\tiny\ttfamily,
    keywordstyle=\color{blue},
    commentstyle=\color{gray},
    stringstyle=\color{red},
    numbers=left,
    numberstyle=\tiny,
    stepnumber=1,
    numbersep=5pt,
    backgroundcolor=\color{gray!10},
    showspaces=false,
    showstringspaces=false,
    showtabs=false,
    frame=single,
    tabsize=2,
    captionpos=b,
    breaklines=true,
    breakatwhitespace=false,
    escapeinside={\%*}{*)}
}

% Informations du document
\title[OAuth 2.0 avec Keycloak]{Projet 2: Implémentation OAuth 2.0 dans une Application Web}
\subtitle{Mise en place d'un serveur d'autorisation avec Keycloak}
\author{Projet Sécurité}
\institute{Université}
\date{\today}

\begin{document}

% Page de titre
\begin{frame}
    \titlepage
\end{frame}

% Table des matières
\begin{frame}
    \frametitle{Plan de la Présentation}
    \tableofcontents
\end{frame}

\section{Introduction à OAuth 2.0}

\begin{frame}
    \frametitle{Qu'est-ce qu'OAuth 2.0?}
    \begin{block}{Définition}
        OAuth 2.0 est un \textbf{framework d'autorisation} qui permet à une application d'obtenir un accès limité aux ressources d'un utilisateur sur un service HTTP.
    \end{block}
    
    \begin{columns}
        \begin{column}{0.5\textwidth}
            \textbf{Problème résolu:}
            \begin{itemize}
                \item Éviter le partage de mots de passe
                \item Contrôle granulaire des permissions
                \item Révocabilité des accès
                \item Sécurité renforcée
            \end{itemize}
        \end{column}
        \begin{column}{0.5\textwidth}
            \textbf{Avantages:}
            \begin{itemize}
                \item Standard industriel
                \item Interopérabilité
                \item Flexibilité
                \item Sécurité éprouvée
            \end{itemize}
        \end{column}
    \end{columns}
\end{frame}

\begin{frame}
    \frametitle{Acteurs OAuth 2.0}
    \begin{center}
        \begin{tikzpicture}[scale=0.8]
            % Resource Owner
            \node[draw, rectangle, fill=blue!20, minimum width=2cm, minimum height=1cm] (ro) at (0,3) {Resource Owner\\Utilisateur};
            
            % Client
            \node[draw, rectangle, fill=green!20, minimum width=2cm, minimum height=1cm] (client) at (0,0) {Client\\Application Web};
            
            % Authorization Server
            \node[draw, rectangle, fill=red!20, minimum width=2cm, minimum height=1cm] (auth) at (4,3) {Authorization Server\\Keycloak};
            
            % Resource Server
            \node[draw, rectangle, fill=yellow!20, minimum width=2cm, minimum height=1cm] (resource) at (4,0) {Resource Server\\API Protégée};
            
            % Arrows
            \draw[->, thick] (ro) -- (auth) node[midway, above] {1. Autorisation};
            \draw[->, thick] (client) -- (auth) node[midway, right] {2. Demande token};
            \draw[->, thick] (auth) -- (client) node[midway, right] {3. Token d'accès};
            \draw[->, thick] (client) -- (resource) node[midway, below] {4. Accès ressource};
        \end{tikzpicture}
    \end{center}
\end{frame}

\begin{frame}
    \frametitle{Flux d'Autorisation OAuth 2.0}
    \begin{enumerate}
        \item \textbf{Redirection vers l'autorisation:} L'application redirige l'utilisateur vers le serveur d'autorisation
        \item \textbf{Authentification:} L'utilisateur s'authentifie et autorise l'application
        \item \textbf{Code d'autorisation:} Le serveur renvoie un code d'autorisation
        \item \textbf{Échange de token:} L'application échange le code contre un token d'accès
        \item \textbf{Accès aux ressources:} L'application utilise le token pour accéder aux ressources
    \end{enumerate}
    
    \begin{alertblock}{Sécurité}
        Le code d'autorisation est échangé contre le token d'accès côté serveur, jamais côté client.
    \end{alertblock}
\end{frame}

\section{Keycloak: Serveur d'Autorisation}

\begin{frame}
    \frametitle{Présentation de Keycloak}
    \begin{block}{Keycloak}
        Keycloak est un \textbf{serveur d'authentification et d'autorisation} open-source qui implémente les standards OAuth 2.0, OpenID Connect et SAML.
    \end{block}
    
    \begin{columns}
        \begin{column}{0.5\textwidth}
            \textbf{Fonctionnalités:}
            \begin{itemize}
                \item Authentification unique (SSO)
                \item Gestion des utilisateurs
                \item Gestion des rôles et permissions
                \item Intégration LDAP/AD
                \item Interface d'administration
                \item API REST complète
            \end{itemize}
        \end{column}
        \begin{column}{0.5\textwidth}
            \textbf{Avantages:}
            \begin{itemize}
                \item Open source
                \item Standards conformes
                \item Haute disponibilité
                \item Scalabilité
                \item Documentation complète
                \item Communauté active
            \end{itemize}
        \end{column}
    \end{columns}
\end{frame}

\begin{frame}
    \frametitle{Architecture Keycloak}
    \begin{center}
        \begin{tikzpicture}[scale=0.7]
            % Keycloak Server
            \node[draw, rectangle, fill=keycloakblue!20, minimum width=3cm, minimum height=2cm] (keycloak) at (0,0) {Keycloak Server};
            
            % Components
            \node[draw, rectangle, fill=oauthgreen!20, minimum width=2cm, minimum height=0.8cm] (auth) at (-2,1.5) {Auth Server};
            \node[draw, rectangle, fill=oauthgreen!20, minimum width=2cm, minimum height=0.8cm] (token) at (2,1.5) {Token Manager};
            \node[draw, rectangle, fill=oauthgreen!20, minimum width=2cm, minimum height=0.8cm] (user) at (-2,0) {User Management};
            \node[draw, rectangle, fill=oauthgreen!20, minimum width=2cm, minimum height=0.8cm] (admin) at (2,0) {Admin Console};
            \node[draw, rectangle, fill=oauthgreen!20, minimum width=2cm, minimum height=0.8cm] (realm) at (0,-1.5) {Realm Management};
            
            % Database
            \node[draw, rectangle, fill=gray!20, minimum width=2cm, minimum height=1cm] (db) at (0,-3) {Base de données\\PostgreSQL};
            
            % Connections
            \draw[<->] (keycloak) -- (auth);
            \draw[<->] (keycloak) -- (token);
            \draw[<->] (keycloak) -- (user);
            \draw[<->] (keycloak) -- (admin);
            \draw[<->] (keycloak) -- (realm);
            \draw[<->] (keycloak) -- (db);
        \end{tikzpicture}
    \end{center}
\end{frame}

\section{Implémentation Pratique}

\begin{frame}
    \frametitle{Architecture du Projet}
    \begin{center}
        \begin{tikzpicture}[scale=0.6]
            % User
            \node[draw, circle, fill=blue!20] (user) at (0,4) {Utilisateur};
            
            % Web App
            \node[draw, rectangle, fill=green!20, minimum width=2.5cm, minimum height=1.5cm] (webapp) at (0,2) {Application Web\\Node.js + Express};
            
            % Keycloak
            \node[draw, rectangle, fill=keycloakblue!20, minimum width=2.5cm, minimum height=1.5cm] (keycloak) at (4,2) {Keycloak\\Serveur d'Autorisation};
            
            % Database
            \node[draw, rectangle, fill=gray!20, minimum width=2.5cm, minimum height=1.5cm] (db) at (4,0) {PostgreSQL\\Base de données};
            
            % API
            \node[draw, rectangle, fill=yellow!20, minimum width=2.5cm, minimum height=1.5cm] (api) at (0,0) {API Protégée\\Ressources};
            
            % Arrows
            \draw[<->, thick] (user) -- (webapp);
            \draw[<->, thick] (webapp) -- (keycloak);
            \draw[<->, thick] (keycloak) -- (db);
            \draw[<->, thick] (webapp) -- (api);
        \end{tikzpicture}
    \end{center}
\end{frame}

\begin{frame}[fragile]
    \frametitle{Configuration Docker Compose}
    \begin{lstlisting}[language=bash, caption=Configuration des services]
version: '3.8'

services:
  keycloak:
    image: quay.io/keycloak/keycloak:22.0
    environment:
      KEYCLOAK_ADMIN: admin
      KEYCLOAK_ADMIN_PASSWORD: admin123
      KC_DB: postgres
      KC_DB_URL: jdbc:postgresql://postgres:5432/keycloak
    ports:
      - "8080:8080"
    command: start-dev

  postgres:
    image: postgres:15
    environment:
      POSTGRES_DB: keycloak
      POSTGRES_USER: keycloak
      POSTGRES_PASSWORD: keycloak123

  web-app:
    build: ./web-app
    ports:
      - "3000:3000"
    environment:
      KEYCLOAK_URL: http://keycloak:8080
      KEYCLOAK_REALM: oauth-demo
      KEYCLOAK_CLIENT_ID: web-app-client
    \end{lstlisting}
\end{frame}

\begin{frame}[fragile]
    \frametitle{Configuration OAuth 2.0 dans l'Application}
    \begin{lstlisting}[language=JavaScript, caption=Configuration Passport.js]
const { Issuer, Strategy: OpenIDConnectStrategy } = require('openid-client');

// Découverte du serveur Keycloak
const keycloakIssuer = await Issuer.discover(
  `${KEYCLOAK_URL}/realms/${KEYCLOAK_REALM}`
);

// Configuration du client
const client = new keycloakIssuer.Client({
  client_id: KEYCLOAK_CLIENT_ID,
  client_secret: KEYCLOAK_CLIENT_SECRET,
  redirect_uris: ['http://localhost:3000/auth/callback'],
  response_types: ['code'],
});

// Configuration de la stratégie Passport
passport.use('oidc', new OpenIDConnectStrategy({
  client: client,
  params: { scope: 'openid email profile' }
}, (tokenset, userinfo, done) => {
  return done(null, userinfo);
}));
    \end{lstlisting}
\end{frame}

\begin{frame}[fragile]
    \frametitle{Routes d'Authentification}
    \begin{lstlisting}[language=JavaScript, caption=Routes Express.js]
// Route de connexion
app.get('/login', passport.authenticate('oidc'));

// Callback OAuth 2.0
app.get('/auth/callback', 
  passport.authenticate('oidc', { failureRedirect: '/error' }),
  (req, res) => {
    res.redirect('/dashboard');
  }
);

// Route protégée
app.get('/api/protected', (req, res) => {
  if (!req.isAuthenticated()) {
    return res.status(401).json({ error: 'Unauthorized' });
  }
  
  res.json({
    message: 'This is a protected resource',
    user: req.user,
    timestamp: new Date().toISOString()
  });
});

// Déconnexion
app.get('/logout', (req, res) => {
  req.logout((err) => {
    if (err) console.error('Logout error:', err);
    res.redirect('/');
  });
});
    \end{lstlisting}
\end{frame}

\section{Sécurité et Bonnes Pratiques}

\begin{frame}
    \frametitle{Considérations de Sécurité}
    \begin{block}{Sécurité OAuth 2.0}
        \begin{itemize}
            \item \textbf{HTTPS obligatoire} en production
            \item \textbf{Validation des tokens} côté serveur
            \item \textbf{Expiration des tokens} configurée
            \item \textbf{Refresh tokens} pour la continuité
            \item \textbf{Scopes limités} aux besoins réels
        \end{itemize}
    \end{block}
    
    \begin{alertblock}{Risques à éviter}
        \begin{itemize}
            \item Stockage des tokens côté client
            \item Transmission en HTTP
            \item Tokens avec durée de vie excessive
            \item Scopes trop permissifs
            \item Gestion d'erreurs révélatrice
        \end{itemize}
    \end{alertblock}
\end{frame}

\begin{frame}
    \frametitle{Configuration Sécurisée Keycloak}
    \begin{columns}
        \begin{column}{0.5\textwidth}
            \textbf{Paramètres de sécurité:}
            \begin{itemize}
                \item Durée de vie des tokens
                \item Politique de mots de passe
                \item Authentification à deux facteurs
                \item Audit des connexions
                \item Rate limiting
                \item CORS configuré
            \end{itemize}
        \end{column}
        \begin{column}{0.5\textwidth}
            \textbf{Monitoring:}
            \begin{itemize}
                \item Logs d'authentification
                \item Métriques de performance
                \item Alertes de sécurité
                \item Tableau de bord admin
                \item Rapports d'audit
                \item Intégration SIEM
            \end{itemize}
        \end{column}
    \end{columns}
\end{frame}

\section{Démonstration}

\begin{frame}
    \frametitle{Fonctionnalités Implémentées}
    \begin{block}{Application Web}
        \begin{itemize}
            \item \textbf{Page d'accueil} avec présentation du projet
            \item \textbf{Authentification OAuth 2.0} via Keycloak
            \item \textbf{Dashboard utilisateur} avec informations de session
            \item \textbf{Profil utilisateur} avec détails du token JWT
            \item \textbf{API protégée} démontrant l'utilisation des tokens
            \item \textbf{Gestion des erreurs} et messages utilisateur
        \end{itemize}
    \end{block}
    
    \begin{block}{Interface Utilisateur}
        \begin{itemize}
            \item Design responsive avec Bootstrap 5
            \item Interface en français
            \item Navigation intuitive
            \item Feedback visuel des actions
            \item Gestion des états d'authentification
        \end{itemize}
    \end{block}
\end{frame}

\begin{frame}
    \frametitle{Flux d'Utilisation}
    \begin{enumerate}
        \item \textbf{Accès à l'application} sur \texttt{http://localhost:3000}
        \item \textbf{Connexion} via le bouton "Se connecter avec OAuth 2.0"
        \item \textbf{Redirection vers Keycloak} pour l'authentification
        \item \textbf{Authentification} avec les identifiants de test
        \item \textbf{Retour à l'application} avec session active
        \item \textbf{Accès au dashboard} et aux ressources protégées
        \item \textbf{Consultation du profil} avec détails du token JWT
        \item \textbf{Test de l'API} protégée via interface web
        \item \textbf{Déconnexion} et retour à l'état initial
    \end{enumerate}
\end{frame}

\section{Conclusion}

\begin{frame}
    \frametitle{Points Clés du Projet}
    \begin{block}{Objectifs Atteints}
        \begin{itemize}
            \item \textbf{Compréhension théorique} d'OAuth 2.0 et de ses flux
            \item \textbf{Implémentation pratique} avec Keycloak
            \item \textbf{Application web fonctionnelle} avec authentification
            \item \textbf{Sécurité respectée} selon les bonnes pratiques
            \item \textbf{Documentation complète} du projet
        \end{itemize}
    \end{block}
    
    \begin{block}{Compétences Développées}
        \begin{itemize}
            \item Configuration et administration de Keycloak
            \item Intégration OAuth 2.0 dans une application Node.js
            \item Gestion des sessions et tokens JWT
            \item Développement d'interfaces utilisateur sécurisées
            \item Containerisation avec Docker
        \end{itemize}
    \end{block}
\end{frame}

\begin{frame}
    \frametitle{Perspectives d'Évolution}
    \begin{block}{Améliorations Possibles}
        \begin{itemize}
            \item \textbf{Multi-tenancy} avec plusieurs realms
            \item \textbf{Intégration LDAP/AD} pour l'authentification
            \item \textbf{Authentification à deux facteurs} (2FA)
            \item \textbf{API Gateway} pour la gestion centralisée
            \item \textbf{Monitoring et observabilité} avancés
            \item \textbf{Tests automatisés} et CI/CD
        \end{itemize}
    \end{block}
    
    \begin{block}{Cas d'Usage Réels}
        \begin{itemize}
            \item Applications d'entreprise avec SSO
            \item APIs microservices sécurisées
            \item Portails clients avec authentification fédérée
            \item Applications mobiles avec OAuth 2.0
        \end{itemize}
    \end{block}
\end{frame}

\begin{frame}
    \frametitle{Questions et Discussion}
    \begin{center}
        \Huge Merci pour votre attention!
        
        \vspace{1cm}
        
        \Large Questions et Discussion
        
        \vspace{1cm}
        
        \normalsize
        \textbf{Projet disponible sur:} \url{https://github.com/votre-repo/oauth-keycloak-demo}
        
        \textbf{Documentation:} README.md avec instructions de déploiement
    \end{center}
\end{frame}

\end{document}
